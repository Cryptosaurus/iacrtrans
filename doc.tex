\documentclass[preprint]{iacrtrans}
\usepackage[utf8]{inputenc}

\author{Gaëtan Leurent\inst{1} \and Alice\inst{2} \and Bob\inst{2}}
\institute{Inria, France, \email{gaetan.leurent@inria.fr} \and ACME}
\title{\publname}
\subtitle{\LaTeX{} Class Documentation}

\begin{document}

\maketitle

\keywords{\publname \and FSE \and \LaTeX}
\begin{abstract}
  This document is a quick introduction to the \LaTeX{} class for the
  \publname journal.
\end{abstract}

\section*{Introduction}

The \texttt{iacrtans} \LaTeX{} class will be used by the new
``\publname'' journal.  The class is based on standard \LaTeX{}
classes and packages (mainly the \texttt{article} class with
\texttt{amsmath}), and should be similar in use to the \texttt{llncs}
class used by Springer's proceedings.  The \LaTeX{} source of this
documentation is meant as an example to show basic usage of the class.

Since this is the zero-th issue of the journal, the class is still in
developpement and feedback and comments are welcome.

\subsection*{Packages used}

The class is based on the standard \texttt{article} class, and loads
the following packages:
\begin{itemize}
\item \texttt{geometry}, \texttt{secsty}, \texttt{fancyhdr}, \texttt{mathtools},
  \texttt{float}, \texttt{microtype}
\item \texttt{amsmath}, \texttt{amssymb}, \texttt{amsthm}
\item \texttt{hyperref}, \texttt{etoolbox}, \texttt{xcolor} (unless
  the \texttt{[nohyperref]} option is used)
\item \texttt{lineno} (in \texttt{[submission]} mode)
\end{itemize}

\section{Publication type}

The class supports three publication types, selected with the
following class options
\begin{itemize}
\item \texttt{[final]} for final papers (default mode)
\item \texttt{[preprint]} for preprint (without copyright info)
\item \texttt{[submission]} for submissions (anonymous)
\item \texttt{[draft]} is similar to preprint, but activates draft
  mode for other packages (\emph{e.g.} \texttt{gaphicx, tikz})
\end{itemize}

\section{Other Class Options}

The \texttt{[spthm]} option provide theorem environments that emulates
\texttt{llncs} class's \texttt{sptheorem}, while the
\texttt{iacrtrans} class is based on \AmS{} theorem.
\begin{itemize}
\item A \texttt{\textbackslash{}spnewtheorem} wrapper is provided
  around \AmS{} \texttt{\textbackslash{}newtheorem}.  Note that the
  styling option are ignored, and you should use standard
  \texttt{amsthm} command for fine control.
\item The \AmS{} theorem environments will not automatically add a
  QED symbol at the end of proofs.
\end{itemize}

The \texttt{[nohyperref]} option disables the automatic loading of
\texttt{hyperref}.  Use this is if your document fails to compile with
\texttt{hyperref} for some reason.

\section{Commands}

\subsection{Title, authors, affiliations, ...}

Author names and affiliations should be seperated by the
\texttt{\textbackslash{}and} macro.




\subsection{Theorems}




\section{Further instructions}

\paragraph{Floats}

Float captions should be below the float.  The \texttt{float} package
loaded by the class should take of this automatically.

\paragraph{Pictures}

We recommend to use the \texttt{tikz} package to render pictures.

\paragraph{Algorithms}

We recommend the \texttt{algorithmcx} package.

\paragraph{Bibliography}

\end{document}
